% This is samplepaper.tex, a sample chapter demonstrating the
% LLNCS macro package for Springer Computer Science proceedings;
% Version 2.20 of 2017/10/04
%
\documentclass[runningheads]{llncs}
%
\usepackage{graphicx}
\usepackage{amssymb,amsmath} 

% Used for displaying a sample figure. If possible, figure files should
% be included in EPS format.
%
% If you use the hyperref package, please uncomment the following line
% to display URLs in blue roman font according to Springer's eBook style:
% \renewcommand\UrlFont{\color{blue}\rmfamily}

\usepackage[draft,inline]{fixme}

\usepackage{pygmentex}

\FXRegisterAuthor{wh}{awh}{WH}%Wes
\FXRegisterAuthor{cn}{acn}{CN}%Chase
\FXRegisterAuthor{ep}{aep}{EP}%Eric

\newcommand{\whnt}[1]{\whnote[inline,marginclue]{\textcolor{brickred}{#1}}} 
\newcommand{\cnnt}[1]{\ydnote[inline,marginclue]{\textcolor{cobalt}{#1}}} 
\newcommand{\epnt}[1]{\epnote[inline,marginclue]{\textcolor{cobalt}{#1}}} 


\begin{document}
%
\title{Voting Theory in the Lean Theorem Prover}

%\thanks{Supported by organization x.}
%
%\titlerunning{Abbreviated paper title}
% If the paper title is too long for the running head, you can set
% an abbreviated paper title here
%
\author{First Author\inst{1}\orcidID{0000-1111-2222-3333} \and
Second Author\inst{2,3}\orcidID{1111-2222-3333-4444} \and
Third Author\inst{3}\orcidID{2222--3333-4444-5555}}
%
\authorrunning{F. Author et al.}
% First names are abbreviated in the running head.
% If there are more than two authors, 'et al.' is used.
%
\institute{Princeton University, Princeton NJ 08544, USA \and
Springer Heidelberg, Tiergartenstr. 17, 69121 Heidelberg, Germany
\email{lncs@springer.com}\\
\url{http://www.springer.com/gp/computer-science/lncs} \and
ABC Institute, Rupert-Karls-University Heidelberg, Heidelberg, Germany\\
\email{\{abc,lncs\}@uni-heidelberg.de}}
%
\maketitle              % typeset the header of the contribution
%
\begin{abstract}
The abstract should briefly summarize the contents of the paper in
150--250 words.

\keywords{voting theory  \and social choice theory \and interactive theorem proving \and Lean}
\end{abstract}
%
%
%

\section{Introduction}

\section{Framework}

In this section, we define the basic objects of voting theory: profiles, social choice correspondences, etc. We first give standard set-theoretic definitions and then their type-theoretic counterparts in Lean syntax.

For our set-theoretic definitions, we fix infinite sets $\mathcal{V}$ and $\mathcal{X}$ of voters and candidates, respectively. Given $X\subseteq\mathcal{X}$, let $\mathcal{B}(X)$ be the set of all binary relations on $X$. To better match our Lean formalization, we take a binary relation on $X$ to be a function $S: X\times X \to \{0,1\}$. Given $x,y\in X$, we write `$xSy$' for $S(x,y)=1$.

%\begin{definition} \textnormal{A \emph{profile} is a function $\mathbf{Q}:V\to \mathcal{B}(X)$ for some finite $V\subseteq\mathcal{V}$ and $X\subseteq\mathcal{X}$. Let $\mathsf{Prof}$ be the set of all profiles.}
%\end{definition}

\begin{definition}\label{ProfileDef} \textnormal{For $V\subseteq\mathcal{V}$ and $X\subseteq\mathcal{X}$, a \emph{$(V,X)$-profile} is a map $\mathbf{Q}:V\to \mathcal{B}(X)$. We write `$\mathbf{Q}_i$' for the relation $\mathbf{Q}(i)$. Given a $(V,X)$-profile $\mathbf{Q}$, let $V(\mathbf{Q})$ be $V$ and $X(\mathbf{Q})$  be $X$. We then define a function $\mathsf{Prof}$ that assigns to each pair $(V,X)$ of $V\subseteq\mathcal{V}$ and $X\subseteq\mathcal{X}$ the set $\mathsf{Prof}(V,X)$ of all $(V,X)$-profiles. Finally, define \[\mathsf{PROF} = \underset{V\subseteq\mathcal{V},X\subseteq\mathcal{X}}{\bigcup}\mathsf{Prof}(V,X).\]}\end{definition}


%\textnormal{A \textit{profile} is a $(V,X)$-profile for some $V\subseteq\mathcal{V}$ and $X\subseteq\mathcal{X}$.}

%\textnormal{Let $\mathsf{Prof}(V,X)$ be the set of all $(V,X)$-profiles and $\mathsf{Prof}$ the set of all profiles.}

%\whnote{One problem with the above definition is that we cannot extract $X$ from $\mathbf{Q}$, because, e.g., $\mathcal{B}(X)$ may be empty.... In our previous papers, where $\mathbf{Q}(i)$ had to be a linear order, we can extract $X$ from $\mathbf{Q}$---provided there is more than one candidate. If there is only one candidate for $\mathbf{Q}$,  then we cannot extract $X$ from $\mathbf{Q}$....}



Depending on the application, one can interpret $x\mathbf{Q}_i y$ to mean either (i) that voter $i$ strictly prefers $x$ to $y$ or (ii) that voter $i$ strictly prefers $x$ to $y$ or is indifferent between $x$ and $y$. Under interpretation (i), we use `$\mathbf{P}$' for a profile; under interpretation (ii), we use `$\mathbf{R}$' for a profile.\footnote{\label{VoterNote}Approach (ii) is more general, since it allows one to distinguish between voter $i$ being \textit{indifferent} between $x$ and $y$, defined as $x\mathbf{R}_iy$ and $y\mathbf{R}_ix$, vs. $x$ and $y$ being \textit{noncomparable} for $i$, defined as \textit{neither} $x\mathbf{R}_iy$ \textit{nor} $y\mathbf{R}_ix$. When the distinction between voter indifference and noncomparability is not needed, approach (i) can be simpler.} A profile $\mathbf{Q}$ is said to be \emph{asymmetric} (\emph{transitive}, etc.) if for every $i\in V$, $\mathbf{Q}_i$ is asymmetric (transitive, etc.). Of course, asymmetric profiles only make sense under interpretation (i), whereas under interpretation (ii), profiles should be reflexive.

To translate Definition \ref{ProfileDef} into Lean, we first think of $V$ and $X$ as types, rather than sets, and then represent the function \textsf{Prof} from Definition \ref{ProfileDef} as follows:\footnote{When writing type expressions, arrows associate to the right, so, e.g., the expression  `\texttt{V $\to$ X $\to$ X $\to$ Prop}' stands for \texttt{V $\to$ (X $\to$ (X $\to$ Prop))}.}
%\begin{itemize}
%\item[] \texttt{def Prof: Type 0 $\to$ Type 0 $\to$ Type 0 := 
%\item[] \qquad\qquad\,\,\, $\lambda$ (V X : Type), V $\to$ X $\to$ X $\to$ Prop}
%\end{itemize}
\begin{itemize}
\item[] \texttt{def Prof := $\lambda$ (V X : Type), V $\to$ X $\to$ X $\to$ Prop}
\end{itemize}
Here \texttt{Prop} is a special type that plays the role of $\{0,1\}$ in the formalization of binary relations mentioned above.  The definition states that \texttt{Prof} is a function that given two types, \texttt{V} and \texttt{X}, outputs the type \texttt{V $\to$ X $\to$ X $\to$ Prop}. Because \texttt{X $\to$ X $\to$ Prop} is the type of binary relations on \texttt{X}, an inhabitant of the type \texttt{V $\to$ X $\to$ X $\to$ Prop} can be viewed as a $(V,X)$-profile. Thus, we may think of \texttt{Prof V X} as the type of $(V,X)$-profiles.

\begin{definition} Margin...
\end{definition}


% whose type is \texttt{Type $\to$ Type $\to$ Type}, 


Next we define two kinds of functions that take profiles as inputs. The first, called a \textit{social choice correspondences} (SCC), assign to a given profile a set of candidates, who are considered tied for winning the election. It is common to consider ``domain restrictions'' on the set of profiles for which the SCC is defined. Thus, one may define an SCC as a function on some set $\mathcal{D}$ of profiles such that for all $\mathbf{Q}\in\mathcal{D}$, we have ${\emptyset\neq F(\mathbf{Q})\subseteq X(\mathbf{Q})}$. However, for our purposes, it is more convenient to use the following equivalent approach.

%\begin{definition} \textnormal{A \emph{voting method} is a function $F$ on $\mathsf{Prof}$ such that for each $\mathbf{Q}\in\mathsf{Prof}$, we have $F(\mathbf{Q})\subseteq X(\mathbf{Q})$. We abuse terminology and call the set $\{\mathbf{Q}\mid F(\mathbf{Q})\neq\emptyset\}$ the \emph{domain} of the voting method.}
%\end{definition}

%\whnote{Here's another approach, which seems closer to what we do with dependent types in Lean.}


\begin{definition} \textnormal{For $V\subseteq\mathcal{V}$ and $X\subseteq\mathcal{X}$, a \textit{social choice correspondence for $(V,X)$}, or $(V,X)$-SCC, is a function  $F: \mathsf{Prof}(V,X)\to \wp(X)$. We abuse terminology and call the set $\{\mathbf{Q}\in\mathsf{Prof}(V,X)\mid F(\mathbf{Q})\neq\emptyset \}$ the \textit{domain} of the $(V,X)$-SCC.}

\textnormal{Let $\mathsf{SCC}$ be a function that assigns to each pair $(V,X)$ of $V\subseteq\mathcal{V}$ and $X\subseteq\mathcal{X}$ the set of all $(V,X)$-SCCs.}
\end{definition}

We represent the function $\mathsf{SCC}$ in Lean as follows, where \texttt{set X} is the type of subsets of \texttt{X}:\footnote{When writing type expressions, function application binds more strongly than arrow, so `\texttt{Prof V X $\to$ set X}' stands for \texttt{(Prof V X) $\to$ set X}.}
\begin{itemize}
\item[] \texttt{def SCC := $\lambda$ (V X : Type), Prof V X $\to$ set X}
\end{itemize}
The definition states that \texttt{SCC} is a function that given two types, \texttt{V} and \texttt{X}, outputs the type \texttt{Prof V X $\to$ set X}, which is the type of $(V,X)$-SCCs.

\begin{example} Given a profile $\mathbf{P}$ and $x\in X(\mathbf{P})$, we say that \textit{$x$ is a Condorcet winner in $\mathbf{P}$} if $V$ is finite and for each $y\in X(\mathbf{P})\setminus \{x\}$, $Margin_\mathbf{P}(x,y)>0$.  For $(V,X)$, consider the Condorcet SCC for $(V,X)$ defined as follows:
\[\mathrm{Cond}_{(V,X)}(\mathbf{P})=\begin{cases} \{x\} & \mbox{if $x$ is a Condorcet winner in $\mathbf{P}$} \\ X(\mathbf{P}) & \mbox{otherwise}\end{cases}.\]
We represent this $(V,X)$-SCC in Lean as follows:
\begin{itemize}
\item[] ...
\end{itemize}
\end{example}

Most voting methods (e.g., Plurality, Borda, Instant Runoff) are defined not only for a fixed set of voters and candidates but for any set of voters and candidates, which motivates the following definition.

\begin{definition}\label{VSCC} \textnormal{A \textit{variable-election social choice correspondence} (VSCC) is a function $F$ that assigns to each pair $(V,X)$ of a $V\subseteq \mathcal{V}$ and $X\subseteq\mathcal{X}$ a $(V,X)$-SCC. We abuse terminology and call the set $\{\mathbf{Q}\in\mathsf{PROF}\mid F(V(\mathbf{Q}),X(\mathbf{Q}))(\mathbf{Q})\neq\emptyset \}$ the \textit{domain} of the VSCC.}
\end{definition}

\noindent An equivalent but perhaps more intuitive approach would define a VSCC to be a function on $\mathsf{PROF}$ (rather than $\wp(\mathcal{V})\times\wp(\mathcal{X})$) such that for each $\mathbf{Q}\in\mathsf{PROF}$, we have $F(\mathbf{Q})\subseteq X(\mathbf{Q})$;\footnote{This is the definition of a \textit{voting method} used in \cite{} with the additional stipulations that $F(\mathbf{Q})\neq\emptyset$ and  that $V(\mathbf{Q})$ and $X(\mathbf{Q})$ are nonempty and finite.} abusing terminology, we could then call the set $\{\mathbf{Q}\in\mathsf{Prop}\mid F(\mathbf{Q})\neq\emptyset\}$ the \emph{domain} of the VSCC. However, we have presented Definition \ref{VSCC} above because it nicely connects with our formalization in Lean.

%\whnote{That definition of VSCC makes the connection with dependent types obviously.}


In Lean, we define the type of VSCCs as a \textit{dependent function type}:
\begin{itemize}
\item[] \texttt{def VSCC := $\Pi$ (V X : Type), SCC V X.}
\end{itemize}
The definition states that a VSCC is a function that for any types \texttt{V} and \texttt{X} returns a function of the type \texttt{SCC V X}, i.e., a $(V,X)$-SCC.

The second type of function we consider assigns to a given profile a binary relation on the set of candidates in the profile. %We call such functions \textit{variable-election collective choice rules}.\footnote{About CCRs...}  

\begin{definition}\textnormal{For $V\subseteq\mathcal{V}$ and $X\subseteq\mathcal{X}$, a \textit{collective choice rule for $(V,X)$}, or $(V,X)$-CCR, is a function  $f: \mathsf{Prof}(V,X)\to \mathcal{B}(X)$. Let $\mathsf{CCR}$ be a function that assigns to each pair $(V,X)$ of $V\subseteq\mathcal{V}$ and $X\subseteq\mathcal{X}$ the set of all $(V,X)$-CCRs.}\end{definition}

\noindent Depending on the application, one can interpret the binary relation $f(\mathbf{Q})$ in one of two ways: $xf(\mathbf{Q})y$ can mean (a) $x$ defeats $y$ socially or (b) $x$ defeats or is tied with $y$ socially.\footnote{As in Footnote \ref{VoterNote}, approach (b) is more general, since it allows one to distinguish between ``social indifference'' and ``social noncomparability'' (see, e.g., ...). When notions of social indifference and noncomparability are not needed, approach (a) can be simpler.} Once again, there is also the issue of ``domain restrictions.'' Under approach $(a)$, we can mark that the CCR is ``undefined'' on a profile $\mathbf{Q}$ by setting $f(\mathbf{Q})= X(\mathbf{Q})\times X(\mathbf{Q})$. Then we can abuse terminology and call $\{\mathbf{Q}\in\mathsf{Prof}(V,X)\mid f(\mathbf{Q})\neq X(\mathbf{Q})\times X(\mathbf{Q}) \}$ the domain of $f$. Under approach (b), we can mark that the CCR is ``undefined'' on $\mathbf{Q}$ by setting $f(\mathbf{Q})=\emptyset$. Then we can abuse terminology and call $\{\mathbf{Q}\in\mathsf{Prop}(V,X)\mid f(\mathbf{Q})\neq \emptyset\}$ the domain of~$f$. 

A CCR $f$ is said to be \textit{asymmetric} (resp.~\textit{transitive}, etc.), if for all $\mathbf{Q}$ in the domain of $f$, $f(\mathbf{Q})$ is asymmetric (transitive, etc.). Of course, asymmetric CCRs only make sense under interpretation (a) above, whereas under interpretation (b), CCRs should be reflexive.

In Lean, our representation of the function $\mathsf{CCR}$ is similar to that of $\mathsf{SCC}$:
\begin{itemize}
\item[] \texttt{def CCR := $\lambda$ (V X : Type), Prof V X $\to$ X $\to$ X $\to$ Prop}
\end{itemize}
Once again, we can consider functions that are not restricted to a fixed set of voters and candidates.

\begin{definition}\label{VCCR} \textnormal{A \emph{variable-election collective choice rule} (VCCR) is a function that assigns to each pair $(V,X)$ of a $V\subseteq\mathcal{V}$ and $X\subseteq\mathcal{X}$ a $(V,X)$-CCR.}% such that for all $\mathbf{Q}\in\mathsf{Prof}$, $f(\mathbf{Q})$ is a binary relation on $X(\mathbf{Q})$.}
\end{definition}

\noindent An equivalent but perhaps more intuitive definition takes a VCCR to be a function $f$ on $\mathsf{Prof}$ (instead of $\mathcal{V}\times\mathcal{X}$) such that for all $\mathbf{Q}\in\mathsf{Prof}$, $f(\mathbf{Q})$ is a binary relation on $X(\mathbf{Q})$.\footnote{This is the definition of a VCCR used in \cite{} with the aditional stipulation that $V(\mathbf{Q})$ and $X(\mathbf{Q})$ are nonempty and finite.} However, we have presented Definition \ref{VCCR} above because it nicely connects with our formalization in Lean, which as before defines the type of VCCRs to be a dependent function type:
\begin{itemize}
\item[] \texttt{def VCCR := $\Pi$ (V X : Type), CCR V X.}
\end{itemize}
The definition states that a VCCR is a function that for any types \texttt{V} and \texttt{X} returns a function of the type \texttt{CCR V X}, i.e., a $(V,X)$-CCR.

\newpage


\whnote{Let's think about how to formalize the following...}

\begin{definition} \textnormal{Given $X\subseteq\mathcal{X}$, a \textit{choice function on $X$} is a function that assigns to each nonempty $Y\subseteq X$ a nonempty $C(Y)\subseteq Y$.}
\end{definition}

\begin{definition}\textnormal{For $V\subseteq\mathcal{V}$ and $X\subseteq\mathcal{X}$, a \textit{functional collective choice rule for $(V,X)$}, or \textit{$(V,X)$-FCCR}, is a function $\mathbb{F}$ that assigns to each $\mathbf{Q}\in  \mathsf{Prof}(V,X)$ a choice function on $X$.}\end{definition}

\begin{definition}\label{VCCR} \textnormal{A \emph{variable-election functional collective choice rule} (VFCCR) is a function $f$ that assigns to each pair $(V,X)$ of a $V\subseteq\mathcal{V}$ and $X\subseteq\mathcal{X}$ a $(V,X)$-FCCR.}% such that for all $\mathbf{Q}\in\mathsf{Prof}$, $f(\mathbf{Q})$ is a binary relation on $X(\mathbf{Q})$.}
\end{definition}

\section{Theorems}

\section{Discussion}

\section{Conclusion}




\newpage

\section{First Section}
\subsection{A Subsection Sample}
Please note that the first paragraph of a section or subsection is
not indented. The first paragraph that follows a table, figure,
equation etc. does not need an indent, either.

Subsequent paragraphs, however, are indented.

\subsubsection{Sample Heading (Third Level)} Only two levels of
headings should be numbered. Lower level headings remain unnumbered;
they are formatted as run-in headings.

\paragraph{Sample Heading (Fourth Level)}
The contribution should contain no more than four levels of
headings. Table~\ref{tab1} gives a summary of all heading levels.

\begin{table}
\caption{Table captions should be placed above the
tables.}\label{tab1}
\begin{tabular}{|l|l|l|}
\hline
Heading level &  Example & Font size and style\\
\hline
Title (centered) &  {\Large\bfseries Lecture Notes} & 14 point, bold\\
1st-level heading &  {\large\bfseries 1 Introduction} & 12 point, bold\\
2nd-level heading & {\bfseries 2.1 Printing Area} & 10 point, bold\\
3rd-level heading & {\bfseries Run-in Heading in Bold.} Text follows & 10 point, bold\\
4th-level heading & {\itshape Lowest Level Heading.} Text follows & 10 point, italic\\
\hline
\end{tabular}
\end{table}


\noindent Displayed equations are centered and set on a separate
line.
\begin{equation}
x + y = z
\end{equation}
Please try to avoid rasterized images for line-art diagrams and
schemas. Whenever possible, use vector graphics instead (see
Fig.~\ref{fig1}).

\begin{figure}
\includegraphics[width=\textwidth]{fig1.eps}
\caption{A figure caption is always placed below the illustration.
Please note that short captions are centered, while long ones are
justified by the macro package automatically.} \label{fig1}
\end{figure}

\begin{theorem}
This is a sample theorem. The run-in heading is set in bold, while
the following text appears in italics. Definitions, lemmas,
propositions, and corollaries are styled the same way.
\end{theorem}
%
% the environments 'definition', 'lemma', 'proposition', 'corollary',
% 'remark', and 'example' are defined in the LLNCS documentclass as well.
%
\begin{proof}
Proofs, examples, and remarks have the initial word in italics,
while the following text appears in normal font.
\end{proof}
For citations of references, we prefer the use of square brackets
and consecutive numbers. Citations using labels or the author/year
convention are also acceptable. The following bibliography provides
a sample reference list with entries for journal
articles~\cite{ref_article1}, an LNCS chapter~\cite{ref_lncs1}, a
book~\cite{ref_book1}, proceedings without editors~\cite{ref_proc1},
and a homepage~\cite{ref_url1}. Multiple citations are grouped
\cite{ref_article1,ref_lncs1,ref_book1},
\cite{ref_article1,ref_book1,ref_proc1,ref_url1}.
%
% ---- Bibliography ----
%
% BibTeX users should specify bibliography style 'splncs04'.
% References will then be sorted and formatted in the correct style.
%
% \bibliographystyle{splncs04}
% \bibliography{mybibliography}
%
\begin{thebibliography}{8}
\bibitem{ref_article1}
Author, F.: Article title. Journal \textbf{2}(5), 99--110 (2016)

\bibitem{ref_lncs1}
Author, F., Author, S.: Title of a proceedings paper. In: Editor,
F., Editor, S. (eds.) CONFERENCE 2016, LNCS, vol. 9999, pp. 1--13.
Springer, Heidelberg (2016). \doi{10.10007/1234567890}

\bibitem{ref_book1}
Author, F., Author, S., Author, T.: Book title. 2nd edn. Publisher,
Location (1999)

\bibitem{ref_proc1}
Author, A.-B.: Contribution title. In: 9th International Proceedings
on Proceedings, pp. 1--2. Publisher, Location (2010)

\bibitem{ref_url1}
LNCS Homepage, \url{http://www.springer.com/lncs}. Last accessed 4
Oct 2017
\end{thebibliography}
\end{document}
