% This is samplepaper.tex, a sample chapter demonstrating the
% LLNCS macro package for Springer Computer Science proceedings;
% Version 2.20 of 2017/10/04
%
\documentclass[runningheads]{llncs}
%
\usepackage{graphicx}
%\usepackage{amsmath,amsthm,amssymb} 

% Used for displaying a sample figure. If possible, figure files should
% be included in EPS format.
%
% If you use the hyperref package, please uncomment the following line
% to display URLs in blue roman font according to Springer's eBook style:
% \renewcommand\UrlFont{\color{blue}\rmfamily}


\begin{document}
%
\title{Voting Theory in the LEAN Theorem Prover\thanks{Supported by organization x.}}
%
%\titlerunning{Abbreviated paper title}
% If the paper title is too long for the running head, you can set
% an abbreviated paper title here
%
\author{First Author\inst{1}\orcidID{0000-1111-2222-3333} \and
Second Author\inst{2,3}\orcidID{1111-2222-3333-4444} \and
Third Author\inst{3}\orcidID{2222--3333-4444-5555}}
%
\authorrunning{F. Author et al.}
% First names are abbreviated in the running head.
% If there are more than two authors, 'et al.' is used.
%
\institute{Princeton University, Princeton NJ 08544, USA \and
Springer Heidelberg, Tiergartenstr. 17, 69121 Heidelberg, Germany
\email{lncs@springer.com}\\
\url{http://www.springer.com/gp/computer-science/lncs} \and
ABC Institute, Rupert-Karls-University Heidelberg, Heidelberg, Germany\\
\email{\{abc,lncs\}@uni-heidelberg.de}}
%
\maketitle              % typeset the header of the contribution
%
\begin{abstract}
The abstract should briefly summarize the contents of the paper in
150--250 words.

\keywords{First keyword  \and Second keyword \and Another keyword.}
\end{abstract}
%
%
%

\section{Introduction}

\section{Framework}

For the following definition, we fix infinite sets $\mathcal{V}$ and $\mathcal{X}$ of voters and candidates, respectively. Given $X\subseteq\mathcal{X}$, let $\mathcal{B}(X)$ be the set of all binary relations on $X$. Given a binary relation $S$ on $X$ and $x,y\in X$, we write `$xSy$' for $(x,y)\in S$.

\begin{definition} \textnormal{A \emph{profile} is a function $\mathbf{Q}:V\to \mathcal{B}(X)$ for some finite $V\subseteq\mathcal{V}$ and $X\subseteq\mathcal{X}$. Let $\mathsf{Prof}$ be the set of all profiles.}
\end{definition}

Depending on the application, one can interpret $x\mathbf{Q}_i y$ to mean either (i) that voter $i$ strictly prefers $x$ to $y$ or (ii) that voter $i$ strictly prefers $x$ to $y$ or is indifferent between $x$ and $y$. Under interpretation (i), we use `$\mathbf{P}$' for a profile; under interpretation (ii), we use `$\mathbf{R}$' for a profile.\footnote{\label{VoterNote}Approach (ii) is more general, since it allows one to distinguish between voter $i$ being \textit{indifferent} between $x$ and $y$, defined as $x\mathbf{R}_iy$ and $y\mathbf{R}_ix$, vs. $x$ and $y$ being \textit{noncomparable} for $i$, defined as \textit{neither} $x\mathbf{R}_iy$ \textit{nor} $y\mathbf{R}_ix$. When the distinction between voter indifference and noncomparability is not needed, approach (i) can be simpler.} A profile $\mathbf{Q}$ is said to be \emph{asymmetric} (\emph{transitive}, etc.) if for every $i\in V$, $\mathbf{Q}_i$ is asymmetric (transitive, etc.). Of course, asymmetric profiles only make sense under interpretation (i), whereas under interpretation (ii), profiles should be reflexive.

In LEAN, ...


Next we define two kinds of functions that take profiles as inputs. The first, which we call \textit{voting methods} (also known as \textit{social choice correspondences}), assign to a given profile a set of candidates, who are considered tied for winning the election. It is common to consider ``domain restrictions'' on the set of profiles for which the voting method is defined. Thus, one may define a voting method as a function on some set $\mathcal{D}$ of profiles such that for all $\mathbf{Q}\in\mathcal{D}$, we have ${\emptyset\neq F(\mathbf{Q})\subseteq X(\mathbf{Q})}$. However, for our purposes, it is more convenient to use the following equivalent approach.

\begin{definition} \textnormal{A \emph{voting method} is a function $F$ on $\mathsf{Prof}$ such that for each $\mathbf{Q}\in\mathsf{Prof}$, we have $F(\mathbf{Q})\subseteq X(\mathbf{Q})$. We abuse terminology and call the set $\{\mathbf{Q}\mid F(\mathbf{Q})\neq\emptyset\}$ the \emph{domain} of the voting method.}
\end{definition}

In LEAN, ...

The second type of function we consider assigns to a given profile a binary relation on the set of candidates in the profile. We call such functions \textit{variable-election collective choice rules}.\footnote{About CCRs...}  

\begin{definition} \textnormal{A \emph{variable-election collective choice rule} (VCCR) is a function $f$ on $\mathsf{Prof}$ such that for all $\mathbf{Q}\in\mathsf{Prof}$, $f(\mathbf{Q})$ is a binary relation on $X(\mathbf{Q})$.}
\end{definition}
Depending on the application, one can interpret $xf(\mathbf{Q})y$ as either (a) $x$ defeats $y$ socially or (b) $x$ defeats or is tied with $y$ socially.\footnote{As in Footnote \ref{VoterNote}, approach (b) is more general, since it allows one to distinguish between ``social indifference'' and ``social noncomparability'' (see, e.g., ...). When notions of social indifference and noncomparability are not needed, approach (a) can be simpler.} Once again, there is also the issue of ``domain restrictions.'' Under approach $(a)$, we can mark that the VCCR is ``undefined'' on a profile $\mathbf{Q}$ by setting $f(\mathbf{Q})= X(\mathbf{Q})\times X(\mathbf{Q})$. Then we can abuse terminology and call $\{\mathbf{Q}\mid f(\mathbf{Q})\neq X(\mathbf{Q})\times X(\mathbf{Q}) \}$ the \textit{domain} of $f$. Under approach (b), we can mark that the VCCR is ``undefined'' on $\mathbf{Q}$ by setting $f(\mathbf{Q})=\emptyset$. Then we can abuse terminology and call $\{\mathbf{Q}\mid f(\mathbf{Q})\neq \emptyset\}$ the \textit{domain} of $f$. A VCCR $f$ is said to be \textit{asymmetric} (resp.~\textit{transitive}, etc.), if for all $\mathbf{Q}$ in the domain of $f$, $f(\mathbf{Q})$ is asymmetric (transitive, etc.). Of course, asymmetric VCCRs only make sense under interpretation (a), whereas under interpretation (b), VCCRs should be reflexive.

In LEAN, ...

\section{Theorems}

\section{Discussion}

\section{Conclusion}




\newpage

\section{First Section}
\subsection{A Subsection Sample}
Please note that the first paragraph of a section or subsection is
not indented. The first paragraph that follows a table, figure,
equation etc. does not need an indent, either.

Subsequent paragraphs, however, are indented.

\subsubsection{Sample Heading (Third Level)} Only two levels of
headings should be numbered. Lower level headings remain unnumbered;
they are formatted as run-in headings.

\paragraph{Sample Heading (Fourth Level)}
The contribution should contain no more than four levels of
headings. Table~\ref{tab1} gives a summary of all heading levels.

\begin{table}
\caption{Table captions should be placed above the
tables.}\label{tab1}
\begin{tabular}{|l|l|l|}
\hline
Heading level &  Example & Font size and style\\
\hline
Title (centered) &  {\Large\bfseries Lecture Notes} & 14 point, bold\\
1st-level heading &  {\large\bfseries 1 Introduction} & 12 point, bold\\
2nd-level heading & {\bfseries 2.1 Printing Area} & 10 point, bold\\
3rd-level heading & {\bfseries Run-in Heading in Bold.} Text follows & 10 point, bold\\
4th-level heading & {\itshape Lowest Level Heading.} Text follows & 10 point, italic\\
\hline
\end{tabular}
\end{table}


\noindent Displayed equations are centered and set on a separate
line.
\begin{equation}
x + y = z
\end{equation}
Please try to avoid rasterized images for line-art diagrams and
schemas. Whenever possible, use vector graphics instead (see
Fig.~\ref{fig1}).

\begin{figure}
\includegraphics[width=\textwidth]{fig1.eps}
\caption{A figure caption is always placed below the illustration.
Please note that short captions are centered, while long ones are
justified by the macro package automatically.} \label{fig1}
\end{figure}

\begin{theorem}
This is a sample theorem. The run-in heading is set in bold, while
the following text appears in italics. Definitions, lemmas,
propositions, and corollaries are styled the same way.
\end{theorem}
%
% the environments 'definition', 'lemma', 'proposition', 'corollary',
% 'remark', and 'example' are defined in the LLNCS documentclass as well.
%
\begin{proof}
Proofs, examples, and remarks have the initial word in italics,
while the following text appears in normal font.
\end{proof}
For citations of references, we prefer the use of square brackets
and consecutive numbers. Citations using labels or the author/year
convention are also acceptable. The following bibliography provides
a sample reference list with entries for journal
articles~\cite{ref_article1}, an LNCS chapter~\cite{ref_lncs1}, a
book~\cite{ref_book1}, proceedings without editors~\cite{ref_proc1},
and a homepage~\cite{ref_url1}. Multiple citations are grouped
\cite{ref_article1,ref_lncs1,ref_book1},
\cite{ref_article1,ref_book1,ref_proc1,ref_url1}.
%
% ---- Bibliography ----
%
% BibTeX users should specify bibliography style 'splncs04'.
% References will then be sorted and formatted in the correct style.
%
% \bibliographystyle{splncs04}
% \bibliography{mybibliography}
%
\begin{thebibliography}{8}
\bibitem{ref_article1}
Author, F.: Article title. Journal \textbf{2}(5), 99--110 (2016)

\bibitem{ref_lncs1}
Author, F., Author, S.: Title of a proceedings paper. In: Editor,
F., Editor, S. (eds.) CONFERENCE 2016, LNCS, vol. 9999, pp. 1--13.
Springer, Heidelberg (2016). \doi{10.10007/1234567890}

\bibitem{ref_book1}
Author, F., Author, S., Author, T.: Book title. 2nd edn. Publisher,
Location (1999)

\bibitem{ref_proc1}
Author, A.-B.: Contribution title. In: 9th International Proceedings
on Proceedings, pp. 1--2. Publisher, Location (2010)

\bibitem{ref_url1}
LNCS Homepage, \url{http://www.springer.com/lncs}. Last accessed 4
Oct 2017
\end{thebibliography}
\end{document}
